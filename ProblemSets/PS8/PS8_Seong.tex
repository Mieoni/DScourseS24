\documentclass[12pt, letterpaper]{article}

% Language and font setting
\usepackage[english]{babel}
\usepackage{times}

% Set page size and margins
\usepackage[letterpaper,top=1.5cm,bottom=2cm,left=2cm,right=2cm,marginparwidth=1.75cm]{geometry}
\usepackage[flushmargin,bottom]{footmisc}

% Set line spacing
\usepackage{setspace}
\renewcommand{\baselinestretch}{1.5}

% Set style of section
\usepackage{titlesec}
\titleformat*{\section}{\large\bfseries}
\renewcommand\thesection{Exercise \arabic{section}}

\titlespacing*{\section}{0pt}{1em}{2pt} % Change the spacing before and after the section heading

% Customize the style of article title
\usepackage{titling}
\setlength{\droptitle}{-4em} % reduce the space before title
\settowidth{\thanksmarkwidth}{*}
\setlength{\thanksmargin}{-\thanksmarkwidth} % remove the indent before thanks

% Set shade color for "solutions"
\usepackage{xcolor}
\definecolor{shadecolor}{rgb}{0.92,0.92,0.92}
\usepackage{framed}

% Packages about equations and mathematics
\usepackage{amsmath}
\usepackage{amssymb} % sign of empty set
\usepackage{dutchcal} % set swash words
\usepackage{bbm}

\DeclareMathOperator*{\argmin}{argmin}
\DeclareMathOperator*{\argmax}{argmax}
\DeclareMathOperator*{\plim}{plim}

% Parentheses
\newcommand{\smp}[1]{\left(#1\right)}
\newcommand{\medp}[1]{\left[#1\right]}
\newcommand{\bgp}[1]{\left\{#1\right\}}

% Sign of absolute value and norm
\usepackage{mathtools}
\DeclarePairedDelimiter\abs{\lvert}{\rvert}%
\DeclarePairedDelimiter\norm{\lVert}{\rVert}%
% Swap the definition of \abs* and \norm*, so that \abs and \norm resizes the size of the brackets, and the starred version does not.
\makeatletter
\let\oldabs\abs
\def\abs{\@ifstar{\oldabs}{\oldabs*}}
%
\let\oldnorm\norm
\def\norm{\@ifstar{\oldnorm}{\oldnorm*}}
\makeatother

% Figure
\usepackage{graphicx}
\usepackage{asymptote}
\usepackage{pgfplots}
\pgfplotsset{compat=1.18}
\usetikzlibrary{patterns,decorations.pathreplacing,calligraphy,calc,shapes.misc}

% Table
\usepackage{float} % Fix the placement of tables/figures
\usepackage{booktabs}
\usepackage{siunitx} % align numbers by decimal point
\usepackage{caption}
\usepackage[labelformat=simple]{subcaption} % set captions for subfigure/subtable
\renewcommand\thesubfigure{(\alph{subfigure})}
\renewcommand\thesubtable{(\alph{subtable})}

% Customize the styles of "itemize" and "enumerate"
\usepackage{enumitem}
\setenumerate[1]{itemsep=0pt,partopsep=0pt,parsep=\parskip,topsep=0pt,itemindent=0.5em}
\setitemize[1]{itemsep=0pt,partopsep=0pt,parsep=\parskip,topsep=5pt}
\setdescription{itemsep=0pt,partopsep=0pt,parsep=\parskip,topsep=5pt}


% Code listing
\usepackage{listings}
\lstdefinelanguage{Stata}{
    % System commands
    morekeywords=[1]{bs, clear, display, drop, egen, gen, jackknife, keep, predict, nlcom, reg, reghdfe, replace, use, webuse, xtreg},
    % Keywords
    morekeywords=[2]{forvalues, if, foreach},
    morecomment=[l]{//},
    morecomment=[l]{**},
    morecomment=[s]{/*}{*/},
    % morecomment=[s]{,}{//},
    % The following is used by macros, like `lags'.
    keywordsprefix=[3]\$,
    morecomment=[n][keywordstyle3]{`}{'},
    morestring=[b]",
    sensitive=true,
}


\renewcommand{\ttdefault}{cmtt}
\lstdefinestyle{mystyle}{
  basicstyle=\ttfamily
}

\lstset{style=mystyle,breaklines=true}

\definecolor{structurecolor}{RGB}{60,113,183}
\definecolor{forestgreen}{RGB}{0,153,76}
\definecolor{orchid}{RGB}{225,0,152}
\definecolor{winered}{RGB}{153,0,0}
\lstset{language=Stata,
  % Code design
  commentstyle=\slshape\color{forestgreen},
  identifierstyle=\color{black},
  keywordstyle=[1]{\color{blue}},
  keywordstyle=[2]{\color{blue}},
  keywordstyle=[3]{\color{winered}},
  %otherkeywords={},
  emph={std},
  emphstyle=\color{orchid},
  stringstyle=\color{winered},
  % Line-numbers design
  numbers=left, % where to put the line-numbers      
  % numbersep=5pt,
  numberstyle=\color{gray},
  % Space design
  keepspaces=true, % keep indentation of code
  showspaces=false, % show spaces everywhere adding particular underscores
  showstringspaces=false, % underline spaces within strings only
  showtabs=false,
  % Frame design
  frame=single,
  tabsize=2,
  rulecolor=\color{structurecolor}, % set the frame-color
  framerule=0.2pt,
  % Margin design
  xleftmargin=0.8cm,
  xrightmargin=0.8cm,
  % Punctuation design
  literate={`}{\textasciigrave}1 {~}{{\color{blue}\raisebox{0.5ex}{\texttildelow}}}1,
  upquote=true,
  % Other design
  columns=flexible,
  % backgroundcolor=\color{white},
  % caption=Python example,
  % backgroundcolor=\color{lightgrey},
  % mathescape=false,
}


% Special words
\newcommand{\comp}[1]{{\fontfamily{cmtt}\selectfont\color{blue}#1}}

% Other packages
\usepackage[colorlinks=true, allcolors=blue]{hyperref}



\pagenumbering{gobble}
\title{Econ-5253\\
{Problem Set 8}  \\
\large  \vspace{-1em}}
\author{Mieon Seong}
\date{Due by Apr. 2th, 2024}

\begin{document}
\setlength{\abovedisplayskip}{5pt}
\setlength{\belowdisplayskip}{5pt}
\setlength{\abovedisplayshortskip}{5pt}
\setlength{\belowdisplayshortskip}{5pt}

\maketitle

\subsection*{Question 5 : Using the matrices you just generated, compute $\hat{\beta}_{OLS}$, which is the OLS estimate of $\beta$ using the closed-form solution. How does your estimate compare with the true value of $\beta$ in $(1)$?}
\
\begin{table}[h]
\begin{tabular}{ll}
   & V1         \\
1  & 1.5010518  \\
2  & -1.0008296 \\
3  & -0.2516480 \\
4  & 0.7490406  \\
5  & 3.5005531  \\
6  & -2.0008185 \\
7  & 0.4987148  \\
8  & 1.0028269  \\
9  & 1.2465102  \\
10 & 2.0010012 
\end{tabular}
\end{table}
\\
It is almost the same as the actual the value of $\beta$.
\\
\newpage
\subsection*{Question 6 : Compute $\hat{\beta}_{OLS}$ using gradient descent (as we went over in class). Make sure you appropriately code the gradient vector! Set the "learning rate" (step size) to equal 0.0000003.}
\\
\begin{table}[h]
\begin{tabular}{ll}
            & {[} , 1{]} \\
{[} 1, {]}  & 1.5010518  \\
{[} 2, {]}  & -1.0008296 \\
{[} 3, {]}  & -0.2516480 \\
{[} 4, {]}  & 0.7490406  \\
{[} 5, {]}  & 3.5005531  \\
{[} 6, {]}  & -2.0008185 \\
{[} 7, {]}  & 0.4987148  \\
{[} 8, {]}  & 1.0028269  \\
{[} 9, {]}  & 1.2465102  \\
{[} 10, {]} & 2.0010012 
\end{tabular}
\end{table}
\\
\subsection*{Question 7 : Compute $\hat{\beta}_{OLS}$ using nloptr's L-BFGS algorithm. Do it again using the Nelder-Mead algorithm. Do your answers differ?}
\\
\begin{table}[h]
\begin{tabular}{lll}
   & L-BFGS & Nelder-Mead    \\
1  & 1.4375000 & 1.29909653 \\
2  & -1.0312500 & -1.11422148 \\
3  & -0.1718750 & -0.13531271 \\
4  & 0.8515625 & -0.12435493 \\
5  & 3.5249023 & 3.58933602 \\
6  & -2.0322266 & -2.49532367 \\
7  & 0.5234375 & 0.34502814 \\
8  & 0.8945312 & 0.55734366 \\
9  & 1.3437500 & 0.04687693 \\
10 & 2.0781250 & 1.22608191
\end{tabular}
\end{table}
\\
Yes. It is a little different. 
\\
\newpage
\subsection*{Question 8 : Now Compute $\hat{\beta}_{MLE}$ using nloptr's L-BFGS algorithm.}
\\
\begin{table}[h]
\begin{tabular}{ll}
   &               \\
1  & -3.350324e-07 \\
2  & 1.781591e-07  \\
3  & 1.637747e-07  \\
4  & -8.340968e-08 \\
5  & -6.032341e-07 \\
6  & 4.866040e-07  \\
7  & -6.535479e-08 \\
8  & -6.289788e-08 \\
9  & -2.755593e-07 \\
10 & -2.658273e-07
\end{tabular}
\end{table}
\\
\subsection*{Question 9-1 : Now compute $\hat{\beta}_{OLS}$ the easy way:using lm() and directly calling the matrices Y and X (no need to create a data frame). Make sure you tell lm() not to include the constant! This is done by typing lm(Y ~ X -1)}
\\
\begin{table}[h]
\begin{tabular}{ll}
X1  & 1.5011  \\
X2  & -1.0008 \\
X3  & -0.2516 \\
X4  & 0.7490  \\
X5  & 3.5006  \\
X6  & -2.0008 \\
X7  & 0.4987  \\
X8  & 1.0028  \\
X9  & 1.2465  \\
X10 & 2.0010 
\end{tabular}
\end{table}
\\
\newpage
\subsection*{Question 9-2 : Use modelsummary to export the regression output to a .tex file. In your .tex file, tell me about how similar your estimates of $\hat{\beta}$ are to the "ground truth" $\beta$ that you used to create the data in (1).}
\begin{table}[h]
\begin{tabular}{ll}
X1  & 1.501  \\
X2  & -1.001 \\
X3  & -0.252 \\
X4  & 0.749  \\
X5  & 3.501  \\
X6  & -2.001 \\
X7  & 0.499  \\
X8  & 1.003  \\
X9  & 1.247  \\
X10 & 2.001 
\end{tabular}
\end{table}
My estimates of $\hat{\beta}$ is almost same as the "ground truth" $\beta$.  
\end{document}
